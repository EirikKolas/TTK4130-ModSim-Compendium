\section{Rigid Body Dynamics}

\subsection{Newton-Euler Dynamics}
Here I should add at least book equation (7.90) and (7.93) and explain where the difference comes from in relation to linear acceleration and derivative rules. 


Solving a Newton-Euler problem:
\begin{enumerate}
    \item Point A is CG or fixed
    \begin{itemize}
        \item $\sum \vec{F}=m\vec{a}$
        \item $\sum\vec{T}_{A}=J_A\vec{\dot{\omega}}+\vec{\omega}\times J_A\vec{\omega}$
    \end{itemize}
	  
    \item Point A is arbitrary
    \begin{itemize}
        \item $\sum \vec{F}=m\vec{a}_{CG}$
        \item $\sum\vec{T}_A=J_{CG}\vec{\dot{\omega}}+\vec{\omega}\times J_{CG}\vec{\omega}+\vec{r}_{CG/A}\times m\vec{a}_{CG}$
    \end{itemize}
\end{enumerate}

\subsubsection{Newton-Euler State Space}

The state space for Netwon-Euler generally takes the form,

\begin{equation}
    \begin{bmatrix}
        \mathbf{F} \\ \mathbf{\tau}
    \end{bmatrix} = \begin{bmatrix}
        m\mathbf{I_3} & 0
        \\
        0 & \mathbf{J_{cm}}
    \end{bmatrix} \begin{bmatrix}
        \mathbf{a_{cm}} \\ \mathbf{\alpha}
    \end{bmatrix} + \begin{bmatrix}
        0 \\ \omega\times\mathbf{J_{cm}}\omega
    \end{bmatrix}
\end{equation}

Accounting for that the classic Newton 2nd law equation appears, there is also a term which originates from the definition of torque as function of angular momentum $h$,

\begin{align}
    \begin{split}
        h_{b/c} &= J_{b/c} \cdot \omega_{ib}
        \\
        \tau_{bc} &= \frac{^id}{dt} h_{b/c}
        \\
        &= \frac{^id}{dt}(J_{b/c} \cdot \omega_{ib}) = \frac{^bd}{dt} (J_{b/c} \cdot \omega_{ib}) + \omega_{ib} \times (J_{b/c} \cdot \omega_{ib})
        \\
        &= J_{b/c} \cdot \alpha_{ib} + \omega_{ib} \times (J_{b/c} \cdot \omega_{ib})
    \end{split}
\end{align}

The equation utilizes Equation \eqref{eq:reference_frame_derivative} as we want the equation to be with respect to the body.

\subsection{Linear momentum}
$$
\begin{aligned}
\vec{p}&=m\vec{v}\\
\vec{\dot{p}}&=m\vec{a}=\sum_i\vec{F}_i
\end{aligned}
$$
\subsection{Angular momentum}
$$
\begin{aligned}
\vec{h}_c&= J_c\vec{\omega}\\
\vec{\dot{h}}_c&= J\frac{d}{dt}(\vec{\omega})+\vec{\omega}\times J\vec{\omega}=\sum_i\vec{T}_{c_i}
\end{aligned}
$$
\subsection{Mass center}
The position $\vec{r_c}$ of the center of mass of a rigid body b is defined by

$$
\vec{r}_c=\frac{1}{m}\int_b \vec{r}_p dm
$$
where $m$ is the mass and $\vec{r}_p$ is the position of the mass element $dm$ which is fixed in frame b.
\subsection{Parallel-Axis Theorem}

\subsubsection{Inertia matrix and its elements}

The matrix is given,

\begin{align}
    \begin{split}
        \Vec{h}_c &= J_c^b \omega^b
        \\
        J_c &= \begin{bmatrix}
            J_{xx} & -J_{xy} & -J_{xz}
            \\
            -J_{xy} & J_{yy} & -J_{yz}
            \\
            -J_{xz} & -J_{yz} & J_{zz}
        \end{bmatrix}
    \end{split}
\end{align}

the moments of inerta,

\begin{align}
    \begin{split}
        J_{xx} &= \iiint (y^2 + z^2)\, dm
        \\
        J_{yy} &= \iiint (x^2 + z^2)\, dm
        \\
        J_{zz} &= \iiint (x^2 + y^2)\, dm
    \end{split}
\end{align}

the products of inertia,

\begin{align}
    \begin{split}
        J_{xy} &= \iiint xy \, dm
        \\
        J_{yz} &= \iiint yz \, dm
        \\
        J_{xz} &= \iiint xz \, dm
    \end{split}
\end{align}

\subsubsection{Inertia at arbitrary point}

For an object with at the point $x,y,z$ with a known inertia, a point $x', y', z'$ has the following inertia,

\begin{align}
    \begin{split}
        J_{x'x'} &= J_{xx} + m((y' - y)^2 + (z' - z)^2)
        \\
        J_{y'y'} &= J_{yy} + m((x' - x)^2 + (z' - z)^2)
        \\
        J_{z'z'} &= J_{xx} + m((x' - x)^2 + (y' - y)^2)
    \end{split}
\end{align}

Note: This is given that we are calculating both points in the same frame.

\subsubsection{Change in center of mass}

The parallel axis theorem is given by the following.

\begin{equation}
\label{eq:p_axis}
    M^b_{b/o}=M^b_{b/c}-m(r^b_g)^\times(r^b_g)^\times
\end{equation}

Where $M^b_{b/o}$ is the moment of inertia (MOI) about the new point, o, $M^b_{b/c}$ is MOI about centre of mass, c, and $r^b_g$ is the vector from o to c. 

In some cases you are given an existing MOI $M^b_{b/c}$ which is the MOI about the current centre of mass. You are then told to add a new mass, and calculate the new MOI about the new centre of mass, $M^{'b}_{b/c'}$. This is done using parallel-axis theorem. 

The standard procedure is the following: 
\begin{enumerate}
    \item Calculate what will be the new centre of mass, $r^b_{c'}$,  using \newline $r^b_{c'}= \frac{1}{M}\sum\limits_{i} m_ir^b_i$,
    where $M=\sum\limits_{i} m_i$. Treat the old mass belonging to the old $M^b_{b/c}$ as a point mass at $r^b_{c}$.
    \item You now need to calculate the old MOI's contribution to the new MOI. You do this using the parallel axis theorem, where $o$ is the new center of mass, and $c$ is the old one. $r^b_g = r^b_{oc} = r^b_{c'c}$. 
    \item You now have $M^b_{b/o} = M^b_{b/c'}$, the contribution of the old mass. This must be summed with the contribution of the new masses $m_i$. Note the minus in the formula, it is there due to algebraic tricks. The formula is taken from equation \ref{eq:MOI}.
    \item $M^{'b}_{b/c'} = M^b_{b/c'} + \sum\limits_{i} -m_i(r^b_{c'i})^\times(r^b_{c'i})^\times$
\end{enumerate}

Notation \\
	- Old cg: c \\
	- New cg: o \\
	- Original object: mass and symbol $m$ \\
	- Added (point) mass: mass and symbol $m'$ \\
1. Calculate new inertia matrix around new cg for original object \\
	- Define vector $r$ from new cg to old cg: $r = - \frac{m'}{m' + m} \cdot p$ \\
		- $p:$ vector from old cg to added point mass \\
	- PAT: $J_{m/o} = J_{m/c}-mr^{\times}r^{\times} = J_{m/c}+ m(rr^TI-rr^T)$  \\
		- $J_{m/o}:$ inertia matrix for mass m around new cg c \\
2. Calculate inertia around new cg for added point mass \\
	- Define vector from added mass $m'$ to new cg: $s = \frac{m}{m+m'} \cdot p$ \\
	- PAT: $J_{m'/o} =m'\cdot s^{\times} s^{\times}$ \\
3. Add inertia to gain the total inertia for the object with added point mass: \\
	- $J_{m+m'/o}=J_{m/o}-m'\cdot s^{\times} s^{\times}= J_{m/c}+ m(rr^TI-rr^T) + m'(ss^TI-ss^T)$ \\
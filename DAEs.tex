\section{Differential Algebraic Equations (DAEs)}

Definition:
An implicit set of equations $F\left(\dot{x},x,u\right)=0$ is DAE if $\frac{\partial F}{\partial x}$ is rank deficient along the trajectory $\dot{x},x,u,$ i.e., if
$$
    \frac{\partial F}{\partial\dot{\mathrm{states}}}=\left[\begin{matrix}\frac{\partial F}{\partial\dot{x}}&\frac{\partial F}{\partial\dot{z}}\\\end{matrix}\right]=\left[\begin{matrix}\frac{\partial F}{\partial\dot{x}}&0\\\end{matrix}\right]
$$
Where z is an algebraic state. More simply explained is that some states are not described using their time derivative, but rather an algebraic equation.

\subsection{Fully Implicit vs. Semi-Explicit}
DAEs can be written in fully implicit form as
$$
    F\left(\dot{x},x,z,u\right) = 0
$$
and in semi-explicit form as (usually good for expressing state-space)
\begin{equation*}
    \begin{split}
        \dot x=f(x,z,u) &\rightarrow \text{ Almost ODE (except for z)}\\
        0=g(x,z,u) &\rightarrow \text{ Algebraic equation}
    \end{split}
\end{equation*}

A semi-explicit ODE can be converted to fully implicit like this:
\begin{equation*}
    \begin{split}
        F(\dot{x},x,z,u)=
        \begin{bmatrix}
        \dot{x}-f(x,z,u) \\ g(x,z,u)
        \end{bmatrix}
        =0
    \end{split}
\end{equation*}
And from fully implicit to semi-explicit like this:
\begin{equation*}
    \begin{split}
        \dot x &= v \\
        0 &= F(v,z,x,u)
    \end{split} 
\end{equation*}

\subsection{Well defined DAEs}

$F(\dot x, x, u)$ is well defined if there is a function $\dot{x}=f(x,u)$ such that\\ $F(f(x,u),u)=0$.

% \subsection{Easy and hard DAEs}
% For \textbf{semi-explicit} notation:
% \begin{itemize}
%     \item if $g$ delivers $z$ we can find $\dot x$ and the DAE is easy
%     \item $\Rightarrow$ $\frac{\partial g}{\partial z}$ is full rank 
%     \item $\Rightarrow$ the DAE is of index 1
% \end{itemize}
% then the DAE is easy.

% For \textbf{fully implicit} notation:
% \begin{itemize}
%     \item if $F$ delivers $\dot x$ and $z$ simultaneously
%     \item $\Rightarrow$
%     $[\frac{\partial F}{\partial \dot{x}} \: \: \frac{\partial F}{\partial z} ]$ is full rank. 
%     \item $\Rightarrow$ it is index 1
%     \item $\Rightarrow$ the Implicit Function Theorem holds for $F$ 
% \end{itemize}
% then the DAE is easy.

% Hard DAEs do not "readily" deliver $\dot x$ and $z$.
% But they may nonetheless deliver well-defined trajectories for $x$ and $z$.

\subsection{Differential index}
A DAE's differential index is the number of $\frac{d}{dt}$s needed to turn it into an ODE.

\subsubsection{Index 1 DAEs}
A DAE is of index 1 if one of the below are fulfilled.
\begin{itemize}
    \item $g$ delivers $z$. I.e. $g(x,z,u)=0$ can be solved for $z$.
    \item $\frac{\partial g}{\partial z}$ is full rank
    \item $F$ delivers $\dot x$ and $z$ simultaneously
    \item 
    $
    \begin{bmatrix}
        \frac{\partial F}{\partial \dot{x}} & \frac{\partial F}{\partial z}
    \end{bmatrix}
    $ 
    is full rank. 
        
    \item The implicit function theorem holds for $F$ (See \autoref{section:implicit_function_theorem} on page \pageref{section:implicit_function_theorem}
\end{itemize}
A DAE of index 1 is said to be easy while higher index DAEs are hard.
Hard DAEs do not ”readily” deliver $\dot x$ and $z$. But they may nonetheless deliver well-defined trajectories for $x$ and $z$.

\subsubsection{Index reduction}
Transform DAE of index $n$ to DAE of index $1$ by applying $\frac{d^{n-1}}{dt^{n-1}}$ to the algebraic equations.

\begin{equation}\label{eq:index_reduction}
    \begin{aligned}
        \frac{d}{dt}g(x,z,u)
        &=\frac{\partial g}{\partial{x}}\dot x
        +\frac{\partial g}{\partial{z}}\dot z 
        +\frac{\partial g}{\partial{u}}\dot u =0\\\\
        \dot z &=-\frac{\partial g}{\partial{z}}^{-1}\left( \frac{\partial g}{\partial{x}}f+ \frac{\partial g}{\partial{u}}\dot u \right)
    \end{aligned}
\end{equation}
\begin{enumerate}
    \item Check if DAE is of index 1. If not, ...
    %\item Identify a subset that can be solved for $z$.\\
    %	$g(x,z,u)$ can be divided into equations that give some $z$ and some that are useless 
    \item Apply $\frac{d}{dt}$ using \autoref{eq:index_reduction}.
    \item Substitute $\dot x$ with $f(x,z,u)$ $\Rightarrow$ This leads to new algebraic equations.
    \item Go to step 1.
\end{enumerate}

%Note: Some exam sets just look for using derivatives on the constraint function until one of the derivatives in $\Dot{x} = f(x,z,u)$ appears such that $z$ can be expressed in the constraint.

\subsection{Consistency Conditions}
The reduction of the index of a DAE requires one to collect a set of consistency conditions that need to be enforced in order for the resulting index-1 DAE to match the original one. 

When performing the index reduction, \textbf{collect all the algebraic equations on which a time differentiation is performed}, and add them to the list of consistency conditions.

For example, in the constrained Euler-Lagrange framework, when we get $\ddot c = 0$ in the final state-space formulation, we also need to enforce the consistency conditions $c=0$ and $\dot c = 0$. An practice one can instead use Baumgarte stabilization. (See \autoref{section:Baumgarte} on page \pageref{section:Baumgarte})

\subsection{Tikhonovs Theorem}
Let
\begin{equation*}
    \begin{split}
    \dot x &= f(x,z,u) \\
    \varepsilon\dot z &= g(x,z) 
    \end{split}
\end{equation*}
be a set of ODEs.
If the dynamics $\dot z = g(x, z) $ are stable for all $x$ and $\frac{\partial g}{\partial z}$ is full rank everywhere
$\rightarrow$ $g$ can be used to find $z$

Then 
$$
\lim_{\varepsilon\rightarrow0}(\text{solution of ODE}) = (\text{solution of DAE})
$$
almost everywhere

The theorem is used to simplify a system by removing the fast dynamics.

\subsection{Implicit Function Theorem}\label{section:implicit_function_theorem}
If $F(x,y)$ is smooth and $\frac{\partial}{\partial y}F(x,y)$ is full rank then there exists a function $y=f(x)$ such that $F(x,f(x))=0$ for all $x$.

In other words, the system of implicit equations is solvable for $x$ (at least numerically) when its Jacobian Matrix has full rank and therby explicit.

If this does not hold, the system is a DAE but can be solved even so.

%The theorem can be used to determine if a differential equation is explicit or implicit. Because if the implicit function theorem holds, then
%$$
%\frac{\partial }{\partial \dot{x}}F(\dot{x},x,u) %\Longrightarrow f(x,u)=\dot{x}
%$$


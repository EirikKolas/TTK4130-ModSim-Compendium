\section{Ordinary Differential Equations (ODEs)}

\subsection{Solution of ODEs}
Chapter 1.8 in lecture notes.
\begin{equation}
\label{eq:ODE}
    \dot{x} = f(x)
\end{equation}
\subsubsection{Theorem 1}

Consider the ODE in equation \ref{eq:ODE}. The solution to this equation exists and is unique if equation \ref{eq:strong_solution_condition} holds for some constant $c$. This is known as Lipschitz continuity. 

\begin{equation}
\label{eq:strong_solution_condition}
    ||f(x)-f(y)|| \leq c||x-y||, \forall x, y
\end{equation}

Intuitively this means that the derivative of the function is bounded, e.g. the function $f$ is limited in how fast it can change.
\subsubsection{Theorem 2}
Theorem 2 has weaker requirements, and thus also a weaker "result" (ie. exists vs exists on some interval). 

Given equation \ref{eq:ODE}, if $\frac{\partial f}{\partial x}$ exists and is continuous then the solution to equation \ref{eq:ODE} exists and is continuous on some time interval. 


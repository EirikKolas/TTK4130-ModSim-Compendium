\section{Rotations}
\subsection{Rotation Matrix}
The rotation matrix $\mathbf{R}_b^a$ from $a$ to $b$ has two interpretations:

1. Let the vector $\vec{v}$ have coordinate vector $v^b$ in ref. frame $b$ and coordinate vector $v^a$ in ref. frame $a$. Then the rotation matrix $\mathbf R_b^a$ transforms the coordinate vector in $b$ to the coordinate vector in $a$ according to
$$
v^a=R_b^a v^b
$$
In this equation $\mathbf{R}_b^a$ acts as a coordinate transformation matrix

2. The vector $\vec{p}$ with coordinate vector $\vec{p}$ in ref. frame $a$ is rotated to the vector $\vec{q}$ in \textit{same frame} with coordinate vector $q^a=p^b$ by
$$
q^a=\mathbf{R}_b^a p^a
$$
In this equation $\mathbf R_b^a$ acts as a rotation matrix.

Basically, the difference is what we interpret as the rotating object. In a transformation, the reference frame "rotates" about the vector, such that the same vector is expressed differently in another frame. In a rotation, the vector itself rotates in a reference frame, changing \textit{that} vector's value in the same frame.

\textbf{Properties:}
\begin{itemize}
    \item
    $$
    v^b=R^b_av^a=R^b_aR^a_bv^b
    $$
    \item Inverting the rotation matrix swaps the direction
    $$
    R_a^b=(R_b^a)^{-1}
    $$
    \item Transposing the rotation matrix swaps the direction
    $$
    R_a^b=(R_b^a)^T
    $$
    \item The two statements above give that all rotation matrices are orthogonal (orthonormal) since $R^T=R^{-1}$
    \item All rotation matrices are part of the set
    $$
    SO(3)=\{{R|R\in R^{3\times3}},R^TR=I,\quad \det{(R)}=1\}
    $$
    \item The following is quite useful if you are given a rotation and you want to find the corresponding rotation matrix. 
    $$
    R^b_a = [R^b_ae_1 \quad R^b_ae_2 \quad R^b_ae_3]
    $$
\end{itemize}

\subsection{Euler Angles}
Roll $\phi$, pitch $\theta$ and yaw $\psi$ are defined as rotation around the $x$, $y$ and $z$-axis respectively and have rotation matrices
$$
R_x(\phi)=
\begin{bmatrix}
1 & 0 & 0 \\ 0 & \cos\phi & -\sin\phi \\ 0 & \sin\phi & \cos\phi
\end{bmatrix}
,\;
R_y(\theta)=
\begin{bmatrix}
\cos\theta & 0 & \sin\theta \\ 0 & 1 & 0 \\ -\sin\theta & 0 & \cos\theta
\end{bmatrix}
,\;
R_z(\psi)=
\begin{bmatrix}
\cos\psi & -\sin\psi & 0 \\ \sin\psi & \cos\psi & 0 \\ 0 & 0 & 1
\end{bmatrix}
$$
Any rotation $R$ can be represented with 3 Euler angles.
These represent three principal rotation transformations. The convention is
$$
R=R_z\cdot R_y\cdot R_x
$$

% The 


\subsection{Axis Angle}
Rotation defined using an eigenvector $\vec v$ of the rotation and a parameter $\alpha$. The rotation matrix $R$ can be found as:
$$
R_{v,\alpha}=\cos\alpha \mathbb{I}+\sin(\alpha)(v)^\times+(1-\cos\alpha)vv^T
\qquad v=R_{v,\alpha}v$$
In other literature $\vec v$ gives the direction and the magnitude of the vector $||\vec v||$ gives the angle $\alpha$.



\subsection{Quaternions}
Using the angle $\alpha$ and direction $\vec v$ from the axis-angle representation, the Euler parameters $\eta$ and $\vec\varepsilon$ are given by:
\begin{align}
\eta&=\cos \frac{\alpha}{2}\\
\vec{\varepsilon}&=\vec{v}\sin \frac{\alpha}{2},\quad \mathbb{q}=\begin{bmatrix}
\eta \\ \vec{\varepsilon}
\end{bmatrix}
\end{align}
where $\mathbb q$ is the quaternion.

\textbf{Properties:}

If $||v||=1$ then
$$
||q||^2=\eta^2+\vec{\varepsilon}^T\vec{\varepsilon}=\cos^2 \frac{\alpha}{2}+\sin^2 \frac{\alpha}{2}=1
$$
Using the Euler parameters $\eta$ and $\vec\varepsilon$, the Rotation matrix $R$ can be found as:
$$
R_{\eta,\vec{\varepsilon}}=\mathbb{I}+2\eta {\varepsilon}^\times+2{\varepsilon}^\times{\varepsilon}^\times
$$